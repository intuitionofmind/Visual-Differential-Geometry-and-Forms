\documentclass{article}
\usepackage{amsmath}
\usepackage{graphicx} % Required for inserting images
\usepackage{enumerate}
\usepackage{ctex}
\usepackage{romannum}
\usepackage[colorlinks=true, allcolors=magenta]{hyperref}
\usepackage[legalpaper, margin=100pt]{geometry}
\usepackage{setspace}
\usepackage{tcolorbox}
\renewcommand{\baselinestretch}{1.25}

\setlength{\parindent}{20pt}

\title{第三幕~曲率}
\author{Wayne Zheng}
\date{\today}

\begin{document}

\maketitle
\tableofcontents

\section{平面曲线的曲率}

不存在一维的内蕴曲率概念。
几何与物理的联系:如果一个单位质量的滚珠在一段无摩擦的曲线上以单位速率运动,金属丝就会有一个垂直于切向的作用力作用在滚珠上,牛顿知道,这个力$F$的大小就是曲线的曲率$\kappa$.

曲率最早由牛顿引入,描述曲线的“弯曲性”。
更确切地说,曲率是切线关于弧长的转向率。
如果$\varphi$是切线的变化角,则$\kappa=d\varphi/ds$.
设$\mathbf{T}$和$\mathbf{N}$是曲线的单位切向量和指向曲率中心的单位法向量,则我们容易得到$\delta\mathbf{T}\asymp\mathbf{N}\delta\varphi$,从而
\begin{equation}
    \frac{d\mathbf{T}}{ds}\equiv\mathbf{T}^{\prime}
    =\kappa\mathbf{N}.
\end{equation}
也可以用法向量取代切向量,考虑法向量的转向率
\begin{equation}
    \frac{d\mathbf{N}}{ds}\equiv\mathbf{N}^{\prime}
    =-\kappa\mathbf{T}.
\end{equation}
因为如果将考虑对象从曲线变成曲面,就不存在唯一的切向量了。
换句话说,在平面内,切向量和法向量的转向速率都是曲率$\kappa$, 其变化率的方向是相反的,平行于另一个向量。

一般地,牛顿发现质点在二维平面的运动曲线$[x(t), y(t)]$的曲率公式
\begin{equation}
    \kappa
    =\frac{\dot{x}\ddot{y}-\dot{y}\ddot{x}}{\left(\dot{x}+\dot{y}^{2}\right)^{3/2}}.
\end{equation}
上方的点是关于时间的导数。
对于单位速率运动的质点,$|\mathbf{v}|^{2}=\dot{x}^{2}+\dot{y}^{2}=1$.
如果质点在$\delta{t}$时刻内走过了$\delta\varphi$,此时$\delta{t}=\delta{s}$.
则$\dot{y}$方向的增量为$\delta\dot{y}\asymp \ddot{y}\delta{t}=\ddot{y}\delta{s}$.
根据相似三角形,
\begin{equation*}
    \frac{\delta\varphi}{\ddot{y}\delta{s}}\asymp\frac{1}{\dot{x}}.
\end{equation*}
即
\begin{equation}
    \kappa\equiv\frac{d\varphi}{ds}
    =\frac{\ddot{y}}{\dot{x}}.
\end{equation}
利用三角形另一边的相似性质,容易得此时曲率的另一个表达式$\kappa=-\ddot{x}/\dot{y}$.

\section{三维空间的曲线}

密切平面:
在三维空间中扭曲的曲线,每个无限小的部分仍可以认为是在一个平面内的,这个平面称为\emph{密切平面}(osculating plane)。
密切平面可以看作是曲线在瞬时最贴合的平面。
密切平面由两个(单位)向量切向量$\mathbf{T}$和主法向量$\mathbf{N}$张成。
考虑沿着该曲线的质点运动,切向量$\mathbf{T}$描述的是曲线的瞬时\emph{速度}方向,主法向量$\mathbf{N}$描述的是曲线的瞬时\emph{加速度}方向,指向曲率中心。
密切平面的法向量称为副法向量$\mathbf{B}$.
$(\mathbf{T}, \mathbf{N}, \mathbf{B})$组成了\emph{弗勒内}(Frenet)框架。

设曲线是关于弧长$s$的函数,即$\mathbf{r}=\mathbf{r}(s)$.
则切向量定义为
\begin{equation*}
    \mathbf{T}=\frac{d\mathbf{r}}{ds}.
\end{equation*}
密切平面的旋转速率称为\emph{挠率}(torsion)$\tau$, 也就是副法线绕$\mathbf{T}$的旋转速率,即
\begin{equation}
    \mathbf{B}^{\prime}=-\tau\mathbf{N}.
\end{equation}
在三维空间中,主法线向量$\mathbf{N}$不仅在密切平面内旋转,还要沿着$\mathbf{B}$方向,绕着切向量$\mathbf{T}$旋转,因此综合起来,主法线的变化率为
\begin{equation}
    \mathbf{N}^{\prime}=-\kappa\mathbf{T}+\tau\mathbf{B}.
\end{equation}
因此,三维空间中曲线的\emph{弗勒内-塞雷方程}(Frenet-Serret formulas)总结如下:
\begin{equation}
    \begin{pmatrix}
        \mathbf{T}^{\prime} \\
        \mathbf{N}^{\prime} \\
        \mathbf{B}^{\prime}
    \end{pmatrix}
    =
    \begin{pmatrix}
        0 & \kappa & 0 \\
        -\kappa & 0 & \tau \\
        0 & -\tau & 0
    \end{pmatrix}
    \begin{pmatrix}
        \mathbf{T} \\
        \mathbf{N} \\
        \mathbf{B}
    \end{pmatrix}.
\end{equation}

\section{曲面的主曲率}

选取$p$点作为原点,法线方向为$z$轴,切平面为$x$-$y$平面,建立直角坐标系。
可以用$z=f(x,y)$来表示曲面,在原点处$f(0,0)=0, \partial_{x}f=\partial_{y}f=0$.
当$x, y\rightarrow 0$时,我们可以展开$z\asymp ax^{2}+by^{2}+cxy+dx+ey$.
因为在原点$\partial_{x}f=\partial_{y}f=0$, 所以$\partial_{x}f=2ax+cy+d\rightarrow 0$.
因此$d=e=0$.
我们可以用\emph{非常靠近}且平行于切面面的一个平面$z=k$来切开曲面,得到的截面曲线在原点附近可以表示为
\begin{equation}
    z\asymp ax^{2}+by^{2}+cxy
\end{equation}
是一个二次齐次曲线。
圆锥曲线都有两个相互垂直的对称轴,如果我们选取这两个轴作为$x$轴和$y$轴,则可以消去交叉项,因为在反射变换下$x\rightarrow -x, y\rightarrow y$, $f(x,y)$保持不变。
因此,我们可以将截面曲线表示为
\begin{equation}
    z\asymp ax^{2}+by^{2}.
\end{equation}
根据二维空间中的曲线,我们知道分别沿着$x$轴和$y$轴的曲线可以表示为$z\asymp \frac{1}{2}\kappa_{1}x^{2}$和$z\asymp \frac{1}{2}\kappa_{2}y^{2}$,其中我们定义了$\kappa_{1}=\kappa(0), \kappa_{2}=\kappa(\pi/2)$.
因此,
\begin{equation}
    z\asymp \frac{1}{2}\kappa_{1}x^{2}+\frac{1}{2}\kappa_{2}y^{2}.
\end{equation}
如果考虑任意角度的截面曲线,设该截面与$x$轴的夹角为$\theta$,如果在切平面内沿着$\theta$方向移动一小段距离$\epsilon$,则$x=\epsilon\cos\theta, y=\epsilon\sin\theta$, 则
\begin{equation}
    \kappa(\theta)
    \asymp 2\left(\frac{z}{\epsilon^{2}}\right)
    \asymp 2\left[\frac{\frac{1}{2}\kappa_{1}(\epsilon\cos\theta)^{2}+\frac{1}{2}\kappa_{2}(\epsilon\sin\theta)^{2}}{\epsilon^{2}}\right]
    =\kappa_{1}\cos^{2}\theta+\kappa_{2}\sin^{2}\theta.
\end{equation}
这就是\emph{欧拉曲率公式}。

\section{测地线和测地曲率}

三维空间中,一般曲面内的一般曲线的曲率可以分为两个分量:
在曲面内(对其中的居民可见)的称为\emph{测地曲率}(geodesic curvature)$\kappa_{g}$, 在曲面外(对其中的居民不可见)的称为\emph{法曲率}(normal curvature)$\kappa_{n}$.
如果密切平面垂直于曲面(的法向量),那么所有曲率都是法曲率$\kappa=\kappa_{n}$,$\kappa_{g}=0$.
地球表面的大圆就是这样。

如果质点以单位速率走过一条曲线,其加速度向量的方向指向曲率中心,长度就是曲率,因此可以将质点的加速度称为\emph{曲率向量}(curvature vector)$\overrightarrow{\kappa}=\kappa\mathbf{N}\equiv\dot{\mathbf{T}}$.

设$\mathbf{T}$是曲线$\mathcal{C}$在点$p$处的单位切向量,$\mathbf{n}$是法向量。
$T_{p}$是曲面在点$p$处的切平面,$\mathbf{n}$与$\mathbf{T}$张成法平面$\Pi_{T}$.
$\Pi_{T}$与密切平面(副法向量$\mathbf{B}$)的夹角为$\gamma$, 则\emph{测地曲率向量}$\overrightarrow{\kappa}_{g}$和\emph{法曲率向量}$\overrightarrow{\kappa}_{n}$实际上就是$\overrightarrow{\kappa}$在$T_{p}$和$\Pi_{T}$上的投影分量,即
\begin{equation}
    \kappa_{g}=\kappa\cos\gamma, \quad
    \kappa_{n}=\kappa\sin\gamma.
\end{equation}

默尼耶(Jean Baptiste Meusnier, 法国数学家、物理学家和工程师)在1776年就注意到:
曲面迫使其上面的所有曲线沿着法方向弯曲同等的量,从而给出默尼耶定理:
\begin{tcolorbox}[colback=white, arc=3mm, auto outer arc]
\begin{minipage}[c,t]{1.0\textwidth}
\kaishu
曲面上经过点$p$、指向同一方向$\mathbf{T}$的所有曲线都具有相同的法曲率$\mathbf{\kappa}_{n}(\mathbf{T})$, 即曲面在$\mathbf{T}$方向的法截线曲率。
如果曲线在$p$点密切平面与曲面在$p$点的切平面成夹角$\gamma$, 曲线在$p$点的曲率为$\kappa_{\gamma}$, 
则$\kappa_{\gamma}\sin\gamma=\kappa_{n}(\mathbf{T})$.
\end{minipage}
\end{tcolorbox}

测地线是内蕴的“直线”,其上每一点$\kappa_{g}=0$.
也就是说:
\emph{对于测地线上的每一点,曲面在该点的法向量$\mathbf{n}_{p}$一定位于该点的密切平面$\Pi_{p}$之内}。

%\bibliographystyle{unsrt}
%\bibliography{refs}
\end{document}
