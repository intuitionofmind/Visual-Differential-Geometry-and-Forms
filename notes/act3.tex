\documentclass{article}
\usepackage{amsmath}
\usepackage{graphicx} % Required for inserting images
\usepackage{enumerate}
\usepackage{ctex}
\usepackage{romannum}
\usepackage[colorlinks=true, allcolors=magenta]{hyperref}
\usepackage[legalpaper, margin=100pt]{geometry}
\usepackage{setspace}
\usepackage{tcolorbox}
\renewcommand{\baselinestretch}{1.25}

\setlength{\parindent}{20pt}

\title{第三幕 曲率}
\author{Wayne Zheng}
\date{\today}

\begin{document}

\maketitle
\tableofcontents

\section{平面曲线的曲率}

不存在一维的内蕴曲率概念。
几何与物理的联系:如果一个单位质量的滚珠在一段无摩擦的曲线上以单位速率运动,金属丝就会有一个垂直于切向的作用力作用在滚珠上,牛顿知道,这个力$F$的大小就是曲线的曲率$\kappa$.

曲率最早由牛顿引入,描述曲线的“弯曲性”。
更确切地说,曲率是切线关于弧长的转向率。
如果$\varphi$是切线的变化角,则$\kappa=d\varphi/ds$.
设$\mathbf{T}$和$\mathbf{N}$是曲线的单位切向量和指向曲率中心的单位法向量,则我们容易得到$\delta\mathbf{T}\asymp\mathbf{N}\delta\varphi$,从而
\begin{equation}
    \frac{d\mathbf{T}}{ds}=\kappa\mathbf{N}.
\end{equation}

\section{三维空间的曲线}

密切平面:
在三维空间中扭曲的曲线,每个无限小的部分仍可以认为是在一个平面内的,这个平面称为密切平面(osculating plane)。
密切平面由两个(单位)向量切向量$\mathbf{T}$和主法向量$\mathbf{N}$张成,密切平面的法向量称为副法向量$\mathbf{B}$.
$(\mathbf{T}, \mathbf{N}, \mathbf{B})$组成了\emph{弗勒内}(Frenet)框架.
设曲线是关于弧长$s$的函数,即$\mathbf{r}=\mathbf{r}(s)$.
则切向量定义为
\begin{equation*}
    \mathbf{T}=\frac{d\mathbf{r}}{ds}.
\end{equation*}

%\bibliographystyle{unsrt}
%\bibliography{refs}
\end{document}
