\documentclass{article}
\usepackage{amsmath}
\usepackage{graphicx} % Required for inserting images
\usepackage{enumerate}
\usepackage{ctex}
\usepackage{romannum}
\usepackage[colorlinks=true, allcolors=magenta]{hyperref}
\usepackage[legalpaper, margin=100pt]{geometry}
\usepackage{setspace}
\usepackage{tcolorbox}
\renewcommand{\baselinestretch}{1.25}

\setlength{\parindent}{20pt}

\title{第三幕 曲率}
\author{Wayne Zheng}
\date{\today}

\begin{document}

\maketitle
\tableofcontents

\section{平面曲线的曲率}

不存在一维的内蕴曲率概念。
几何与物理的联系:如果一个单位质量的滚珠在一段无摩擦的曲线上以单位速率运动,金属丝就会有一个垂直于切向的作用力作用在滚珠上,牛顿知道,这个力$F$的大小就是曲线的曲率$\kappa$.

曲率最早由牛顿引入,描述曲线的“弯曲性”。
更确切地说,曲率是切线关于弧长的转向率。
如果$\varphi$是切线的变化角,则$\kappa=d\varphi/ds$.
设$\mathbf{T}$和$\mathbf{N}$是曲线的单位切向量和指向曲率中心的单位法向量,则我们容易得到$\delta\mathbf{T}\asymp\mathbf{N}\delta\varphi$,从而
\begin{equation}
    \frac{d\mathbf{T}}{ds}\equiv\mathbf{T}^{\prime}
    =\kappa\mathbf{N}.
\end{equation}
也可以用法向量取代切向量,考虑法向量的转向率
\begin{equation}
    \frac{d\mathbf{N}}{ds}\equiv\mathbf{N}^{\prime}
    =-\kappa\mathbf{T}.
\end{equation}
因为如果将考虑对象从曲线变成曲面,就不存在唯一的切向量了。
换句话说,在平面内,切向量和法向量的转向速率都是曲率$\kappa$, 其变化率的方向是相反的,平行于另一个向量。

\section{三维空间的曲线}

密切平面:
在三维空间中扭曲的曲线,每个无限小的部分仍可以认为是在一个平面内的,这个平面称为密切平面(osculating plane)。
密切平面可以看作是曲线在瞬时最贴合的平面。
密切平面由两个(单位)向量切向量$\mathbf{T}$和主法向量$\mathbf{N}$张成。
考虑沿着该曲线的质点运动,切向量$\mathbf{T}$描述的是曲线的瞬时\emph{速度}方向,主法向量$\mathbf{N}$描述的是曲线的瞬时\emph{加速度}方向,指向曲率中心。
密切平面的法向量称为副法向量$\mathbf{B}$.
$(\mathbf{T}, \mathbf{N}, \mathbf{B})$组成了\emph{弗勒内}(Frenet)框架。

设曲线是关于弧长$s$的函数,即$\mathbf{r}=\mathbf{r}(s)$.
则切向量定义为
\begin{equation*}
    \mathbf{T}=\frac{d\mathbf{r}}{ds}.
\end{equation*}
密切平面的旋转速率称为\emph{挠率}(torsion)$\tau$, 也就是副法线绕$\mathbf{T}$的旋转速率,即
\begin{equation}
    \mathbf{B}^{\prime}=-\tau\mathbf{N}.
\end{equation}
在三维空间中,主法线向量$\mathbf{N}$不仅在密切平面内旋转,还要沿着$\mathbf{B}$方向,绕着切向量$\mathbf{T}$旋转,因此综合起来,主法线的变化率为
\begin{equation}
    \mathbf{N}^{\prime}=-\kappa\mathbf{T}+\tau\mathbf{B}.
\end{equation}
因此,三维空间中曲线的\emph{弗勒内-塞雷方程}(Frenet-Serret formulas)总结如下:
\begin{equation}
    \begin{pmatrix}
        \mathbf{T}^{\prime} \\
        \mathbf{N}^{\prime} \\
        \mathbf{B}^{\prime}
    \end{pmatrix}
    =
    \begin{pmatrix}
        0 & \kappa & 0 \\
        -\kappa & 0 & \tau \\
        0 & -\tau & 0
    \end{pmatrix}
    \begin{pmatrix}
        \mathbf{T} \\
        \mathbf{N} \\
        \mathbf{B}
    \end{pmatrix}.
\end{equation}


%\bibliographystyle{unsrt}
%\bibliography{refs}
\end{document}
