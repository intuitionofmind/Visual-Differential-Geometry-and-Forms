\documentclass{article}
\usepackage{amsmath}
\usepackage{graphicx} % Required for inserting images
\usepackage{enumerate}
\usepackage{ctex}
\usepackage{romannum}
\usepackage[colorlinks=true, allcolors=magenta]{hyperref}
\usepackage[legalpaper, margin=100pt]{geometry}
\usepackage{setspace}
\usepackage{tcolorbox}
\renewcommand{\baselinestretch}{1.25}

\setlength{\parindent}{20pt}

\title{第5章~伪球面和双曲平面}
\author{Wayne Zheng}
\date{\today}

\begin{document}

\maketitle

贝尔特拉米(Beltrami)是意大利数学家,对双曲几何有重要贡献。
他也发现了拉格朗日力学中的贝尔特拉米等式。
他也知道局部的高斯-博内定理。

\section{利用变分原理证明广义斯涅尔定律}

\begin{tcolorbox}[colback=white, arc=3mm, auto outer arc]
\begin{minipage}[c,t]{1.0\textwidth}
\kaishu
贝尔特拉米等式。
考虑作用量
\begin{equation*}
    S=\int{L}(u, u^{\prime}, x)dx
\end{equation*}
使函数$u(x)$取到极值。
这里$x$是描述系统演化的独立参数(类比于时间$t$),$u$可以看作是广义坐标。
可以定义广义动量$p=\partial{L}/\partial{u^{\prime}}$.
此时变分原理导出的拉格朗日方程简化为
\begin{equation*}
    \frac{dp}{dx}-\frac{\partial{L}}{\partial{u}}=0.
\end{equation*}
哈密顿量$H=u^{\prime}p-L$.
利用拉格朗日方程,则可以得到贝尔特拉米等式:
\begin{equation*}
\begin{aligned}
\frac{dH}{dx}
&=\left(pu^{\prime\prime}+\frac{dp}{dx}u^{\prime}\right)-\left(\frac{\partial{L}}{\partial{u}}u^{\prime}+\frac{\partial{L}}{\partial{u\prime}}u^{\prime\prime}+\frac{\partial{L}}{\partial{x}}\right)=-\frac{\partial{L}}{\partial{x}}.
\end{aligned}
\end{equation*}
这是诺特定理的一个特例。
\end{minipage}
\end{tcolorbox}

考虑光在介质中的传播,坐标是$(x, y)$,路径可以表示成曲线 $y=y(x)$.
不妨进一步假设折射率只是纵坐标的函数$n=n(y)$.
则光从$x_{A}$到到$x_{B}$的作用量为
\begin{equation}
\begin{aligned}
S
&=\int_{x_{A}}^{x_{B}}\frac{ds}{v}=\int_{x_{A}}^{x_{B}}\frac{1}{v(y)}\sqrt{dx^{2}+dy^{2}} \\
&=\int_{x_{A}}^{x_{B}}n(y)\sqrt{1+\left(\frac{dy}{dx}\right)^{2}}dx
\equiv\int_{x_{A}}^{x_{B}}Ldx.
\end{aligned}
\end{equation}
拉氏量$L=n(y)\sqrt{1+y^{\prime{2}}}$.
哈密顿量
\begin{equation}
    H
    =y^{\prime}p-L
    =-\frac{n(y)}{\sqrt{1+y^{\prime{2}}}}.
\end{equation}
根据贝尔特拉米等式,易得$dH/dx=0$,即$H=-C$是一个常数,也就意味着系统沿着$x$方向具有连续的平移不变性,从而对应有一个守恒量,这个守恒量
\begin{equation}
    \frac{n(y)}{\sqrt{1+y^{\prime{2}}}}
    =n(y)\frac{dx}{\sqrt{dx^{2}+dy^{2}}}
    =n(y)\sin\theta=C
\end{equation}
给出了连续介质的斯涅尔定律,$\theta$是光线与入射平面的法线的夹角。
这与几何方法得到的斯涅尔定律一致。

下面讨论两个例子。

%\bibliographystyle{unsrt}
%\bibliography{refs}
\end{document}
